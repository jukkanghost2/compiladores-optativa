\chapter{Conclusión}

Es posible observar que la compilación empleando GCC como CLANG/LLVM respeta las
fases estándar de todo compilador moderno. Las diferencias que se encuentran residen en
los formatos de código interno que utilizan, en especial, código intermedio de \emph{middle-end}:
GIMPLE, para GCC, y LLVM IR, para CLANG/LLVM.
A su vez, podemos observar que el código en \emph{assembler} final es el mismo en esencia,
independiente del compilador usado. Dichas diferencias residen en el nombre de variables y
otros \emph{tags} así como las direcciones de cada elemento (distintas por obvias razones).
CLANG/LLVM permite una mayor flexibilidad a la hora de definir optimizaciones y formatos
de códigos internos a emitir. Sin embargo, para este cometido, no se ha desplegado el
máximo provecho de dichas optimizaciones.