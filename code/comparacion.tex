\chapter{Comparacion entre GCC y CLANG/LLVM}

A partir de un mismo código fuente en lenguaje C, 
se procederá a la compilación empleando tanto GCC como CLANG/LLVM. 
Luego, se llevará a cabo una comparación y breve descripción de lo
observado. Se presentan los códigos relevantes que coinciden 
con las etapas nombradas anteriormente.

\section{Analizador lexico y sintactico}
GCC no permite obtener la salida del analizador lexico, por lo que se analizara 
el arbol abstracto sintactico. 

GCC describe el arbol relacionando sus nodos a partir de punteros con la concatenando 
"@" y un numero. Entonces, es posible armar el arbol siguiendo estos punteros. 
En este fragmento se ve como la declaracion de la variable "a", que es del tipo entera con 
un valor inicial de "20", es traducida al arbol.

\begin{lstlisting}[label=comandoC, caption= Fragmento del codigo fuente del archivo codigo-ejemplo.c. \cite{repositorio}, language=bash]
    int a = 20;      \end{lstlisting}

Se observa en @5 la palabra clave \emph{var\_decl}, la cual indica la declaracion de una variable. En este 
caso, el campo \emph{name} apunta a @9, que es un nodo identificador con el nombre de la variable "a". 
Tambien, se puede obtener el tipo de la variable refiriendose a @10, que muestra \emph{integer\_type}, y lo 
mismo para \emph{init} se refiere al valor inicial apuntado por @12 que muestra que es el valor 20.

\begin{lstlisting}[label=comandoC, caption= Fragmento del arbol de GCC del archivo codigo-ejemplo.c.005t.original. \cite{repositorio}, language=bash]
    @5      var_decl         name: @9       type: @10      scpe: @11   srcp: codigo-ejemplo.c:5      init: @12     
    @9      identifier_node  strg: a        lngt: 1       
    @10     integer_type     name: @24      size: @13      algn: 32      prec: 32       sign: signed   min : @25     max : @26     
    @11     function_decl    name: @27      type: @28      srcp: codigo-ejemplo.c:3      
    @12     integer_cst      type: @10      int: 20    \end{lstlisting}

El arbol construido por CLANG/LLVM se muestra de una forma mas grafica con barras e identacion. 
Se observa una \emph{FunctionDecl}, la cual hace referencia al \emph{main()} y adentro de esta una 
\emph{DeclStmt}, que se encuentra en la linea 5 del codigo fuente. El nodo \emph{VarDecl} 
define la existencia una variable llamada a de tipo \emph{int}, con un valor de 20.

\begin{lstlisting}[label=comandoC, caption= Fragmento del arbol de CLANG/LLVM del archivo ast. \cite{repositorio}, language=bash]
    |-FunctionDecl 0x207e190 <line:3:1, line:17:1> line:3:5 main 'int ()'
    | `-CompoundStmt 0x207ea60 <col:12, line:17:1>
    |   |-DeclStmt 0x207e2d0 <line:5:5, col:15>
    |   | `-VarDecl 0x207e248 <col:5, col:13> col:9 used a 'int' cinit
    |   |   `-IntegerLiteral 0x207e2b0 <col:13> 'int' 20   \end{lstlisting}


\section{Gimple vs LLVM IR}

En esta etapa se comparara como el bucle del codigo fuente se desarrolla en el codigo intermedio 
de ambos compiladores. 

\begin{lstlisting}[label=comandoC, caption= Fragmento del codigo fuente del archivo codigo-ejemplo.c. \cite{repositorio}, language=bash]
    for (int i = 0; i < c; i++)
    {
        arr[i] = suma(a, b);
    }   \end{lstlisting}

GCC emplea \emph{Gimple} para esta etapa que tiene una familiaridad con el lenguaje C, 
en el cual se ve que se crean etiquetas para realizar el control del salto 
y el bucle se transforma en sentencias \emph{if else}. Tambien, se crean variables 
temporales para alojar resultados, como por ejemplo "\_1" para el resultado que 
devuelve la funcion suma.  

\begin{lstlisting}[label=comandoC, caption= Fragmento del \emph{Gimple} de GCC del archivo codigo-ejemplo.c.006t.gimple. \cite{repositorio}, language=bash]
    {
        int i;
        i = 0;
        goto <D.1955>;
        <D.1954>:
        _1 = suma (a, b);
        arr[i] = _1;
        i = i + 1;
        <D.1955>:
        if (i < c) goto <D.1954>; else goto <D.1952>;
        <D.1952>:
    } \end{lstlisting}

Por otro lado se tiene LLVM IR, el cual ya tiene un parecido con el lenguaje \emph{assembler} 
mas que con el lenguaje C. Se utilizan variables temporales para los resultados de las 
instrucciones sin repetir concatenando "\%" y un numero. La etiqueta 10 hace referencia 
al control del salto que en la ultima instruccion decide si saltar a la etiqueta 14 o 24. 
En la etiqueta 14 se cargan las variables \%2 y \%3 que hacen referencia a las variables 
a y b para realizar la llamada a la funcino suma. Después, se guarda el resultado en 
el \emph{array}.

\begin{lstlisting}[label=comandoC, caption= Fragmento del LLVM IR de CLANG/LLVM del archivo llvm-ir.ll. \cite{repositorio}, language=bash]
10:                                               ; preds = %21, %0
  %11 = load i32, i32* %6, align 4
  %12 = load i32, i32* %4, align 4
  %13 = icmp slt i32 %11, %12
  br i1 %13, label %14, label %24

14:                                               ; preds = %10
  %15 = load i32, i32* %2, align 4
  %16 = load i32, i32* %3, align 4
  %17 = call i32 @suma(i32 %15, i32 %16)
  %18 = load i32, i32* %6, align 4
  %19 = sext i32 %18 to i64
  %20 = getelementptr inbounds [10 x i32], [10 x i32]* %5, i64 0, i64 %19
  store i32 %17, i32* %20, align 4
  br label %21

21:                                               ; preds = %14
  %22 = load i32, i32* %6, align 4
  %23 = add nsw i32 %22, 1
  store i32 %23, i32* %6, align 4
  br label %10  \end{lstlisting}

\section{RTL vs MachineInstr}
En esta etapa ambos codigos son poco legibles para humanos por lo que su descripción 
no es sencilla. Se muestra como la declaracion de las variables a y b son expresadas 
por cada compilador. 

\begin{lstlisting}[label=comandoC, caption= Fragmento del codigo fuente del archivo codigo-ejemplo.c. \cite{repositorio}, language=bash]
    int a = 20;
    int b = 30; \end{lstlisting}

\begin{lstlisting}[label=comandoC, caption= Fragmento de RTL de GCC del archivo codigo-ejemplo.c.330r.final. \cite{repositorio}, language=bash]
(insn 6 3 7 2 (set (mem/c:SI (plus:DI (reg/f:DI 6 bp)
                (const_int -60 [0xffffffffffffffc4])) [2 a+0 S4 A32])
        (const_int 20 [0x14])) "codigo-ejemplo.c":5:9 75 {*movsi_internal}
     (nil))
(insn 7 6 8 2 (set (mem/c:SI (plus:DI (reg/f:DI 6 bp)
                (const_int -56 [0xffffffffffffffc8])) [2 b+0 S4 A64])
        (const_int 30 [0x1e])) "codigo-ejemplo.c":6:9 75 {*movsi_internal}
     (nil)) \end{lstlisting}

\begin{lstlisting}[label=comandoC, caption= Fragmento de MachineInstr de CLANG/LLVM del archivo machineinstrs. \cite{repositorio}, language=bash]
    MOV32mi %stack.1, 1, $noreg, 0, $noreg, 20 :: (store 4 into %ir.2)
    MOV32mi %stack.2, 1, $noreg, 0, $noreg, 30 :: (store 4 into %ir.3) \end{lstlisting}


\section{Assembler}
Por ultimo, la salida en codigo \emph{assembler} de ambos compiladores es practicamente la misma. 
A continuacion, se expone la funcion suma expresada en codigo \emph{assembler} destacando 
la similitud entre ambos compiladores aunque los codigos intermedios sean notoriamente 
diferentes en el proceso.

\begin{lstlisting}[label=comandoC, caption= Fragmento del codigo fuente del archivo codigo-ejemplo.c. \cite{repositorio}, language=bash]
int suma (int a, int b) {
    return a + b;
}   \end{lstlisting}

\begin{lstlisting}[label=comandoC, caption= Fragmento del codigo \emph{assembler} de GCC del archivo codigo-ejemplo.s. \cite{repositorio}, language=bash]
    .globl	suma
    .type	suma, @function
suma:
    .LFB1:
    .cfi_startproc
    endbr64
    pushq	%rbp
    .cfi_def_cfa_offset 16
    .cfi_offset 6, -16
    movq	%rsp, %rbp
    .cfi_def_cfa_register 6
    movl	%edi, -4(%rbp)
    movl	%esi, -8(%rbp)
    movl	-4(%rbp), %edx
    movl	-8(%rbp), %eax
    addl	%edx, %eax
    popq	%rbp
    .cfi_def_cfa 7, 8
    ret
    .cfi_endproc
.LFE1:
    .size	suma, .-suma    \end{lstlisting}

\begin{lstlisting}[label=comandoC, caption= Fragmento del codigo \emph{assembler} de CLANG/LLVM del archivo assembly.s. \cite{repositorio}, language=bash]
	.globl	suma                    # -- Begin function suma
	.p2align	4, 0x90
	.type	suma,@function
suma:                                   # @suma
	.cfi_startproc
# %bb.0:
	pushq	%rbp
	.cfi_def_cfa_offset 16
	.cfi_offset %rbp, -16
	movq	%rsp, %rbp
	.cfi_def_cfa_register %rbp
	movl	%edi, -8(%rbp)
	movl	%esi, -4(%rbp)
	movl	-8(%rbp), %eax
	addl	-4(%rbp), %eax
	popq	%rbp
	.cfi_def_cfa %rsp, 8
	retq
.Lfunc_end1:
	.size	suma, .Lfunc_end1-suma
	.cfi_endproc
                                        # -- End function   \end{lstlisting}