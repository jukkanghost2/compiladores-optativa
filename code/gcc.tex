\chapter{Codigo intermedio de GCC}

A continuacion se exponen las distintos codigos intermedios que GCC utiliza en la compilacion. 
Los distintos codigos intermedios estan relacionados en la forma que la salida 
de cada uno es la entrada del siguiente, avanzando desde una representacion general de alto nivel 
hacia una especifica de bajo nivel. 

\section{\emph{Generic}}

\emph{Generic} es un codigo intermedio independiente del lenguaje con estructura de arbol 
que es generado por el \emph{front end}. \emph{Generic} es capaz de representar todos los 
lenguajes admitidos por GCC. \emph{Generic} se produce eliminando construcciones especificas 
del lenguaje del arbol de parseo. 

\section{\emph{Gimple}}

\emph{Gimple} es un codigo intermedio de tres direcciones resultante de desglosar \emph{Generic} en tuplas 
de no mas de tres operandos, a traves de la herramienta interna de GCC llamada \emph{Gimplifier}. 
\emph{Gimple} introduce variables temporales para poder computar expresiones complejas y permite 
supervisar el flujo de control a nivel inferior con sentencia secuenciales y saltos incondicionales. 
\emph{Gimple} es el codigo intermedio principal de GCC (los lenguajes C y C++ se convierten a \emph{Gimple} 
sin pasar por \emph{Generic}), ademas de ser conveniente para optimizar. 

Existen tres tipos de \emph{Gimple}:

\begin{itemize}
    \item \emph{Gimple} de alto nivel que es lo que se obtiene despues de desglosar el \emph{Generic}.
    \item \emph{Gimple} de bajo nivel que se obtiene al linealizar todas las estructuras de flujo de control de 
            del \emph{Gimple} de alto nivel, incluidas las funciones anidadas, el manejo de excepciones y los bucles.
    \item \emph{Gimple} SSA es el \emph{Gimple} de bajo nivel reescrito en la forma SSA.
\end{itemize}

\section{\emph{RTL}}

\emph{RTL} es un codigo intermedio de bajo nivel semejante al lenguaje ensamblador.
