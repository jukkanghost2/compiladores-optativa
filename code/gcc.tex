\chapter{Codigo intermedio de GCC}

A continuacion se exponen las distintos codigos intermedios que GCC utiliza en la compilacion. 
Los distintos codigos intermedios estan relacionados en la forma que la salida 
de cada uno es la entrada del siguiente, avanzando desde una representacion general de alto nivel 
hacia una especifica de bajo nivel. 

\section{Analizador lexico y sintactico}
El analizador lexico lee la secuencia de caracteres desde la salida del preprocesador y agrupa los 
caracteres en secuencias llamadas lexemas.
Por cada lexema, el analizador genera una token con la forma:
\begin{center}
    [\emph{token name, attribute value}]
\end{center}
Donde \emph{token name} son símbolos abstractos usados durante el análisis sintáctico,
y \emph{attribute value} es un puntero a una entrada en la tablas de símbolos.
GCC no permite obtener los tokens válidos en forma de texto. Es un archivo interno.
El analizador sintactico chequea si la gramática del lenguaje acepta la secuencia
de tokens generados por el analizador lexico, sino reporta errores de sintaxis. 
Ademas, el analizador sintactico usa el primer componente de cada token para crear una representación en forma de árbol 
que muestre la estructura de los tokens, es decir, un árbol sintáctico.
Con el \emph{flag} -fdump-tree-original-raw se obtiene la representacion textual del arbol abstracto sintactico.

\begin{lstlisting}[label=comandoC, caption= Comando de compilación del archivo codigo-ejemplo.c \cite{repositorio} para GCC., language=bash]
    $ gcc -fdump-tree-original-raw codigo-ejemplo.c -o codigo-ejemplo  \end{lstlisting}

\section{Analizador semantico}
El analizador semantico utiliza el árbol sintáctico y la información de la tabla de símbolos 
para revisar la consistencia semántica del programa fuente con respecto a la definición del lenguaje. 

\section{\emph{Generic}}

\emph{Generic} es un codigo intermedio independiente del lenguaje con estructura de arbol 
que es generado por el \emph{front end}. \emph{Generic} es capaz de representar todos los 
lenguajes admitidos por GCC. \emph{Generic} se produce eliminando construcciones especificas 
del lenguaje del arbol de parseo. 
GCC pasa del arbol abstracto sintactico a la representacion en \emph{Gimple} en lenguajes como C.

\section{\emph{Gimple}}

\emph{Gimple} es un codigo intermedio de tres direcciones resultante de desglosar \emph{Generic} en tuplas 
de no mas de tres operandos, a traves de la herramienta interna de GCC llamada \emph{Gimplifier}. 
\emph{Gimple} introduce variables temporales para poder computar expresiones complejas y permite 
supervisar el flujo de control a nivel inferior con sentencia secuenciales y saltos incondicionales. 
\emph{Gimple} es el codigo intermedio principal de GCC (los lenguajes C y C++ se convierten a \emph{Gimple} 
sin pasar por \emph{Generic}), ademas de ser conveniente para optimizar. 

Existen tres tipos de \emph{Gimple}:

\begin{itemize}
    \item \emph{Gimple} de alto nivel que es lo que se obtiene despues de desglosar el \emph{Generic}.
    \item \emph{Gimple} de bajo nivel que se obtiene al linealizar todas las estructuras de flujo de control de 
            del \emph{Gimple} de alto nivel, incluidas las funciones anidadas, el manejo de excepciones y los bucles.
    \item \emph{Gimple} SSA es el \emph{Gimple} de bajo nivel reescrito en la forma SSA.
\end{itemize}

Con el \emph{flag} -fdump-tree-gimple se obtiene la representacion en la forma de \emph{Gimple}.

\begin{lstlisting}[label=comandoC, caption= Comando de compilación del archivo codigo-ejemplo.c \cite{repositorio} para GCC., language=bash]
    $ gcc -fdump-tree-gimple codigo-ejemplo.c -o codigo-ejemplo  \end{lstlisting}

Ademas, con el \emph{flag} -fdump-tree-all-graph GCC genera muchos archivos con la extension .cfg 
los cuales pueden visualizarse con una herramienta online Graphviz. Esta herramienta permite 
ver la evolucion del codigo en las distintas pasadas de una manera mucho mas conveniente 
para el usuario. 

\begin{lstlisting}[label=comandoC, caption= Comando de compilación del archivo codigo-ejemplo.c \cite{repositorio} para GCC., language=bash]
    $ gcc -fdump-tree-all-graph codigo-ejemplo.c -o codigo-ejemplo  \end{lstlisting}


\section{\emph{Register Transfer Language}}

\emph{Register Transfer Language} es un codigo intermedio de bajo nivel semejante al lenguaje ensamblador.
La mayor parte del trabajo del compilador se realiza en \emph{Register Transfer Language}. Tiene una forma interna, 
formada por estructuras que apuntan a otras estructuras, y una forma textual 
que se utiliza en la descripción de la máquina y en los volcados de depuración 
impresos. El formulario textual usa paréntesis anidados para indicar los punteros en el formulario interno.
Con el \emph{flag} -fdump-rtl-final se obtiene la representacion en la forma de \emph{Register Transfer Language} ya optimizado por el compilador.

\begin{lstlisting}[label=comandoC, caption= Comando de compilación del archivo codigo-ejemplo.c \cite{repositorio} para GCC., language=bash]
    $ gcc -fdump-rtl-final codigo-ejemplo.c -o codigo-ejemplo  \end{lstlisting}

\section{\emph{Assembler}}
Por ultimo, es posible obtener la salida en \emph{Assembler} con el \emph{flag} -S.

\begin{lstlisting}[label=comandoC, caption= Comando de compilación del archivo codigo-ejemplo.c \cite{repositorio} para GCC., language=bash]
    $ gcc -S codigo-ejemplo.c -o codigo-ejemplo.s  \end{lstlisting}

% \section{Flags de compilacion}
% GCC permite obtener los archivos internos que genera en la compilacion donde se observa 
% el paso del codigo fuente por los codigos intermedios y su optimizacion. Solo se describiran los archivos 
% mas importantes porque GCC genera una gran cantidad de ellos.
% Los nombres por defecto de los archivos se construyen concatenando el nombre base del archivo, el numero de pasada, 
% la letra de la fase y el nombre de la pasada. 

