\chapter{Introduccion}

El codigo intermedio es un codigo interno usado por el compilador para 
representar el codigo fuente. El codigo intermedio esta diseñado para llevar a cabo 
el procesamiento del codigo fuente, como es la optimizacion y la traduccion a codigo maquina. 

Una de las caracteristicas mas esenciales del codigo intermedio es ser 
independiente del \emph{hardware}. Por lo tanto, permite la portabilidad entre distintos 
sistemas.

Otra propiedad importante de todo codigo intermedio es su facil generacion a partir del 
codigo fuente, como asi tambien su facil traduccion al codigo maquina para la arquitectura 
deseada.

No existe un unico codigo intermedio, sino que hay distintos tipos y categorias, variando 
de compilador en compilador. Aunque un mismo compilador puede usar varios tipos de 
codigo intermedio en el proceso.

\section{Tipos de codigos intermedios}

Los distintos codigos intermedios son clasificados en estructurales, lineales o hibridos.

Los codigos intermedios estructurales estan orientados a la forma grafica. Los codigos 
intermedios estructurales son principalmente usados en las etapas iniciales para 
una primera traduccion del codigo fuente. La estructura conformada por nodos y vertices 
llega a ser demasiada grande. Ejemplos de codigos intermedios estructurales son arboles y grafos.

Los codigos intermedios lineales son pseudocodigo para una maquina abstracta por 
lo que tienen varios niveles de abstraccion. Los codigos intermedios lineales son 
simples y compactos, por lo tanto, son convenientes para reorganizar y para optimizar.
Ejemplos de codigos intermedios lineales son RTL, GIMPLE, LLVM IR, etc.

Los codigos intermedios hibridos son una combinacion de los estructurales con los 
lineales, intentando aprovechar las ventajas de cada uno. Ejemplo de codigo intermedio 
hibrido son los grafo de control de flujo.

