\chapter{Introduccion}

El codigo intermedio es un codigo interno usado por el compilador para 
representar el codigo fuente. El codigo intermedio esta diseñado para llevar a cabo 
el procesamiento del codigo fuente, como es la optimizacion y la traduccion a codigo maquina. 

Una de las caracteristicas mas esenciales del codigo intermedio es ser 
independiente del \emph{hardware}. Por lo tanto, permite la portabilidad entre distintos 
sistemas.

Otra propiedad importante de todo codigo intermedio es su facil generacion a partir del 
codigo fuente, como asi tambien su facil traduccion al codigo maquina para la arquitectura 
deseada.

No existe un unico codigo intermedio, sino que hay distintos tipos y categorias, variando 
de compilador en compilador. Aunque un mismo compilador puede usar varios tipos de 
codigo intermedio en el proceso.

A continuacion, se presentan los codigos intermedios utilizados por los compiladores 
GCC y CLANG/LLVM, especificando las caracteristicas de cada uno y comparando sus 
prestaciones posteriormente.
