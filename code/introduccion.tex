\chapter{Introducción}

El código intermedio es un código interno usado por el compilador para 
representar el código fuente. El código intermedio está diseñado para llevar a cabo 
el procesamiento del código fuente, como es la optimización y la traducción a código máquina. 

Una de las características más esenciales del código intermedio es ser 
independiente del \emph{hardware}. Por lo tanto, permite la portabilidad entre distintos 
sistemas.

Otra propiedad importante de todo código intermedio es su fácil generación a partir del 
código fuente, como así también su fácil traducción al código máquina para la arquitectura 
deseada.

No existe un único código intermedio, sino que hay distintos tipos y categorías, variando 
de compilador en compilador. Aunque un mismo compilador puede utilizar varios tipos de 
código intermedio en el proceso\cite{craftinginterpreters}.

A continuación, se presentan los códigos intermedios utilizados por los compiladores 
GCC y CLANG/LLVM, especificando las características de cada uno y comparando sus 
prestaciones posteriormente.

